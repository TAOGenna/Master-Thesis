\chapter{Conclusiones y discusi\'{o}n}
Hemos resuelto un modelo para la cadena de Kitaev en una grilla conectada a un punto cu\'{a}ntico en uno de sus extremos por un t\'{e}rmino de \emph{hopping} y una repulsi\'{o}n coulombiana entre el estado relevante del punto cu\'{a}ntico y el del extremo de la cadena. Repulsiones de rango corto se espera que sean importantes en heteroestructuras realistas de semiconductores-superconductores usadas para crear modos cero de Majorana, pero tratamientos te\'{o}ricos estudiando los efectos de estas interacciones son raros. 

Conforme es variada la energ\'{i}a del estado del punto cu\'{a}ntico, las energ\'{i}as de los dos autoestados del sistema cercanos a cero presentan una de dos formas caracter\'{i}sticas vistas en experimentos y teor\'{i}as previas, se\~{n}alando la presencia de modos cero de Majorana en los extremos de la cadena. En uno de ellos (forma de mo\~{n}o), las energ\'{i}as de los dos estados se cuando la energ\'{i}a del punto cu\'{a}ntico es cercana a la energ\'{i}a de Fermi. En este caso, el acoplamiento entre los MZMs es d\'{e}bil y analizando la funci\'{o}n de onda de estos autoestados, uno ve que un MZM tiene un peso importante en el punto cu\'{a}ntico. Tratando la repulsi\'{o}n coulombiana en la aproximaci\'{o}n irrestricta de Hartree-Fock, entontramos que esta repulsi\'{o}n no afecta esencialmente la calidad de los MZMs.

En contraste, en el otro caso, en el que las energ\'{i}as de los estados de m\'{a}s baja energ\'{i}a como funci\'{o}n del nivel del punto tiene una forma de diamante, se\~{n}alando un acoplamiento m\'{a}s fuerte entre los MZMs (cadenas cortas), el efecto de la repulsi\'{o}n coulombiana interat\'{o}mica es una mayor divisi\'{o}n de los MZMs. Ya que en este caso, la repulsi\'{o}n del MZM en el lado opuesto de la cadena con el estado del punto es relevante, los resultados son consistentes con los de la referencia \cite{PhysRevB.100.144510} los cuales indican, que la interacci\'{o}n entre part\'{i}culas localizadas en sitios distantes juega un papel m\'{a}s destructivo que la interacci\'{o}n entre sitios vecinos m\'{a}s cercanos. 

Pensamos que nuestras conclusiones son tambi\'{e}n validas para modelos m\'{a}s realistas de superconductores topol\'{o}gicos, tal como en referencias \cite{Lutchyn2010MajoranaFA,PhysRevLett.105.177002}, incluso incluyendo subbandas adicionales y el efecto orbital del campo magn\'{e}tico \cite{PhysRevB.107.245423}. Si la forma de diamante del espectro a bajas energ\'{i}as como funci\'{o}n de la energ\'{i}a del punto es experimentalmente obervados, como en la referencia \cite{PhysRevB.98.085125}, esto significa que los MZMs tienen una extensi\'{o}n significativa sobre todo el cable y la repulsi\'{o}n coulombiana en un extremo deteriora a\'{u}n m\'{a}s la calidad de los Majorana. En cambio, la repulsi\'{o}n coulombiana en un extremo no afecta MZMs bien localizados caracter\'{i}stico de una forma de mo\~{n}o en el espectro. 