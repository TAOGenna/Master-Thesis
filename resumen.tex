\begin{resumen}%
En el contexto de una configuraci\'{o}n en el que un punto cu\'{a}ntico es usado como una sonda local para estudiar la calidad de modos de Majorana en un nanocable superconductor, investigamos el efecto que tienen sobre estos la introducci\'{o}n de un t\'{e}rmino de respulsi\'{o}n coulombiana entre el punto y el cable cu\'{a}ntico.\\
En modelos previos, el sistema es descrito por hamiltonianos efectivos a bajas energ\'{i}as que, en su mayor\'{i}a, no toman en cuenta un posible efecto de la repulsi\'{o}n coulombiana. Sin embargo, en el estudio realizado por Ricco \emph{et alia}, los autores usan un modelo que incluye dicha repulsi\'{o}n y afirman que la presencia de este t\'{e}rmino arruina la protecci\'{o}n topol\'{o}gica que poseen los modos cero de Majorana (MZMs).\\
En este trabajo de maestr\'{i}a, sostenemos la tesis de que estudiar el sistema a trav\'{e}s de un hamiltoniano efectivo a bajas energ\'{i}as no captura el verdadero efecto de un t\'{e}rmino de repulsi\'{o}n coulombiana en los MZMs. En lugar de un hamiltoniano efectivo, consideramos t\'{e}rminos de m\'{a}s altas energ\'{i}as y modelamos la repulsi\'{o}n como un t\'{e}rmino de interacci\'{o}n entre el punto y el sitio m\'{a}s cercano a este en el nanocable. Seguidamente, obtenemos num\'{e}ricamente valores de expectaci\'{o}n y el perfil de espectro de energ\'{i}as donde notamos que la \'{u}nica situaci\'{o}n donde los MZMs pierden su protecci\'{o}n topol\'{o}gica es cuando tratamos con un nanocable corto y el espectro de energ\'{i}as presenta un patr\'{o}n de diamante.\\
La presente tesis consta de cuatro cap\'{i}tulos. En el cap\'{i}tulo 1, daremos una breve introducci\'{o}n a los modos de Majorana, cadena de Kitaev y el modelo propuesto por Ricco \emph{et alia}. Luego, en el cap\'{i}tulo 2, introducimos un m\'{e}todo anal\'{i}tico que nos permite conocer la forma matricial de los hamiltonianos que estudiamos y analizamos la cadena de Kitaev aislada. Seguidamente, en el cap\'{i}tulo 3, presentamos los resultados del modelo incluyendo el punto cu\'{a}ntico y el \emph{hopping} y la interacci\'{o}n con el primer sitio de la cadena que sostienen la tesis descrita arriba. Finalmente, en el cap\'{i}tulo 4, mencionamos las conclusiones del presente trabajo de investigaci\'{o}n. 
\begin{center}
    Esta maestr\'{i}a est\'{a} basada en el trabajo cient\'{i}fico: 
    \begin{itemize}
        \item \cite{perez2023effect} R. Kenyi Takagui-Perez, A. Aligia. Effect of interatomic repulsion on majorana zero modes in a coupled quantum-dot-superconducting-nanowire hybrid system, arXiv:2309.10888 [cond-mat.mes-hall], enviado a Physical Review B.
    \end{itemize}
\end{center}
\end{resumen}




%%% Local Variables: 
%%% mode: latex
%%% TeX-master: "template"
%%% End: 
